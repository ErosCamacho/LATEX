 \documentclass[a4paper,12pt]{article}
 \usepackage[latin1]{inputenc}
 \usepackage[T1]{fontenc}
 \usepackage[spanish]{babel}
 
\begin{document}

Ejemplo 1:

\begin{tabular}{|l|c|c|r|}
 \hline
  {\bf Nombre} & {\bf Apellidos} & {\bf DNI} & {\bf Calificaci�n} \\
  \hline \hline
  Juan & L�pez & 43434322L & 5.3 \\
  Emilio & P�rez & 45989845K & 7.2 \\
  Gema & Guti�rrez & 21388383A & 8.1 \\
  \hline
\end{tabular}

Ejemplo 2: 

\begin{tabular}{|l|c|c|r|}
 \hline
  {\bf Nombre}  & {\bf Apellidos} & {\bf DNI} & {\bf Calificaci�n} \\
  \hline
  Juan & L�pez & 43434322L & 5.3 \\
  Emilio & P�rez & 45989845K & 7.2 \\
  Gema & Guti�rez & \multicolumn{2}{l}{ } \\
  \hline
 \end{tabular}
 
Ejemplo 3:

 \begin{tabular}{|@{\bf Cap�tulo: }r@{\qquad Temas: }c@{ - }c|}
 \hline
  1 & 1 & 5 \\
  2 & 6 & 9 \\
  3 & 10 & 14 \\
  \hline
 \end{tabular}
 
Ejemplo 4:

 \begin{tabular}{|p{3cm}|p{1cm}|}
 \hline
   Insertamos este p�rrafo dentro de esta columna de la tabla &
   Y en esta columna insertamos este otro p�rrafo\\
  \hline
 \end{tabular}

\end{document}

\documentclass[a4paper,12pt]{article}
 \usepackage[latin1]{inputenc}
 \usepackage[T1]{fontenc}
 \usepackage[spanish]{babel}
 \usepackage[pdftex]{graphicx}
 \usepackage[vflt]{floatflt}


 \begin{document}
\begin{floatingfigure}{2.5cm}
\includegraphics[width=2.5cm]{coche.jpg}
\caption{El cochecito}
\end{floatingfigure}

Este entorno s�lo funciona si se pone antes de un p�rrafo, la
figura aparecer� lo m�s cerca del lugar en donde se haya escrito,
esto quiere decir que \LaTeX{} primero intenta poner la figura en
la p�gina actual, si no encuentra suficiente espacio vertical
entonces la coloca en la p�gina siguiente.

El argumento [r] es un argumento opcional que hace que el gr�fico
salga a la derecha del texto (no importa lo que se haya puesto al
cargar la librer�a).


 \end{document}

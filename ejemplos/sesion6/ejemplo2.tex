 \documentclass[a4paper,12pt]{article}
 \usepackage[latin1]{inputenc}
 \usepackage[T1]{fontenc}
 \usepackage[spanish]{babel}
 \usepackage{amsmath,amsthm}

 \begin{document}

\section{\LaTeX{} un nuevo sistema de escritura}

Leslie Lamport, en \cite{Lam94}, describe \dots

Como se puede ver en \cite{Lam94,Cas03}, \LaTeX{} es \dots

En \cite[p�gina 45]{Oetiker} tenemos un ejemplo de \dots


\begin{thebibliography}{10}
\bibitem{Lam94}
    L. Lamport.{\em \LaTeX: A Document Preparation System.}
    Addison-Wesley, Reading, Massachusetts, second edition, 1994, ISBN 0-201-52983-1.
\bibitem{Cas03}
    B. Cascales Salina, P. Lucas Saur�n, J.M. Mira Res, A.J. Pallar�s Ru�z, S. S�nchez-Pedre�o.
    {\em El libro de \LaTeX.}
    Pearson-Prentice-Hall, 2003.
\bibitem{Oetiker}
    T.~Oetiker, H.~Partl, I.~Hyna and E.~Schlegl.
    {\em The Not So Short Introduction to \LaTeX $2\varepsilon$}.
\end{thebibliography}

\end{document}

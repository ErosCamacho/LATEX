\documentclass[a4paper,12pt]{article}
\usepackage[latin1]{inputenc}
\usepackage[T1]{fontenc}
\usepackage[spanish]{babel}
\usepackage{amsmath}



\begin{document}
\title{Primero de Fisica}
\author{Eros Camacho Ruiz}
\date{\today}
\maketitle	
\tableofcontents
\vspace{1cm}
\begin{abstract}
	A continuaci�n voy a hacer unas peque�as demostraciones de lo que he podido aprender este a�o. En concreto es El Principio de Bernouilli y la Experiencia de Young.
\end{abstract}
\section{Principio de Bernouilli}
El Principio de Bernouilli es utilizado para el estudio del comportamiento de un fluido en un medio material.
Se refiere a un medio homog�neo y lineal. Todos estos contenidos se encuentran en \cite[Secci�n de mec�nica]{Tipler1}
\subsection{Para una altura variable}
	\begin{equation}
	\label{eq:1}
	P_0+\frac{1}{2}V^2_0\rho+{\rho}gh_0=P_f+\frac{1}{2}V^2_f\rho+{\rho}gh_f
	\end{equation}
\subsection{Para una altura constante}
	\begin{equation}
	\label{eq:2}
		P_0+\frac{1}{2}V^2_0\rho=P_f+\frac{1}{2}V^2_f\rho
	\end{equation}
	\begin{equation}
	\label{eq:3}
		P_0-P_f={\Delta}P=\frac{1}{2}\rho\left(V^2_f-V^2_0\right)
	\end{equation}
	Como puede comprobarse en la Ecuaci�n \ref{eq:3} la ca�da de presi�n depende la velocidad inicial y final del fluido.
\section{Experiencia de Young}
Voy a demostrar la Experiencia de Young, concluiremos diciendo que la intensidad luminosa de un haz de luz s�lo va a depender de la diferencia de caminos que se establezca entre las aberturas. Todos estos contenidos se encuentran en \cite[Secci�n de electromagnetismo]{Tipler2}
Si utilizamos el principio de superposici�n quedar�a:
	\begin{equation}
	\label{eq:4}
\psi_T=\psi_1+\psi_2=
\begin{cases}
\psi_1(r_1)=Acos({\omega}t-kr_1),  \\
\psi_2(r_2)=Acos({\omega}t-kr_2),
\end{cases} 
	\end{equation}
	\begin{equation}
	\psi_T=\psi_1+\psi_2=Acos({\omega}t-kr_1)+Acos({\omega}t-kr_2)
	\end{equation}
	\begin{equation}
	\label{eq:6}
	\psi_T=A(cos({\omega}t-kr_1)+cos({\omega}t-kr_2))
	\end{equation}
Recurriendo a la ecuaci�n \ref{eq:6} y como \dots $cosA+cosB=2cos(\frac{A+B}{2})cos(\frac{A+B}{2})$ la ecuaci�n quedar�a del siguiente modo:
	\begin{equation}
	\label{eq:7}
	\psi_T=2A(cos({\omega}t-\frac{k{\Delta}r}{2})cos(\frac{k{\Delta}r}{2}))
	\end{equation}

Como la intensidad es proporcional a la amplitud al cuadrado la ecuaci�n \ref{eq:7} quedar�a del siguiente modo:
	\begin{equation}
	\label{eq:8}
		I_T=4A^2cos^2\left(\frac{k{\Delta}r}{2}\right)=2A^2\left(1-cos\left(k{\Delta}r\right)\right)
	\end{equation}

Hemos podido demostrar gracias a la ecuaci�n \ref{eq:8} que la intensidad en la experiencia de Young s�lo depende de la diferencia de caminos entre los haces de luz.

\begin{thebibliography}{2}
	\bibitem{Tipler1}
	P. A. Tipler., G. Mosca {\em F��sica para la ciencia y la tecnolog��a. Volumen 1. Mec�nica, oscilaciones y ondas, termodin�mica}
	Revert�, 6�ed. , 2010
	\bibitem{Tipler2}
	P. A. Tipler., G. Mosca {\em F��sica para la ciencia y la tecnolog��a. Volumen 2. Electricidad y magnetismo. Luz}
	 Revert�, 6�ed. , 2010
	
\end{thebibliography}

\end{document}